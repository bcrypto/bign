\section{Выработка и проверка электронной цифровой подписи}\label{SIGN}

\subsection{Входные и выходные данные}

Входными данными алгоритмов ЭЦП являются 
параметры $p$, $a$, $b$, $q$, $G$, которые описывают группу 
точек эллиптической кривой.
Параметры должны удовлетворять условиям алгоритма~\ref{VALEC}. 
По модулю~$p$ определяется уровень стойкости~$l$ как минимальное 
натуральное число, для которого $p<2^{2l}$.

Кроме параметров эллиптической кривой,
входными данными алгоритма выработки ЭЦП
являются сообщение~$X\in\{0,1\}^*$
и личный ключ~$d\in\{1,2,\ldots,q-1\}$.

Выходными данными алгоритма выработки ЭЦП
является слово $S\in\{0,1\}^{3l}$~--- 
подпись~$X$.

Кроме параметров эллиптической кривой,
входными данными алгоритма проверки ЭЦП 
являются сообщение~$X\in\{0,1\}^*$,
подпись~$S\in\{0,1\}^*$ и открытый ключ~$Q\in E_{a,b}^*(\FF_p)$.
%
Открытый ключ~$Q$ должен удовлетворять условиям
алгоритма~\ref{VALPUBKEY}.

Выходными данными алгоритма проверки ЭЦП
является ответ~\texttt{ДА} или~\texttt{НЕТ}.
Ответ \texttt{ДА} означает, что~$S$ 
является корректной подписью~$X$.
Ответ \texttt{НЕТ} означает обратное.

\subsection{Вспомогательные алгоритмы и преобразования, переменные}
\label{SIGN.Aux}

{\bf Алгоритм~\algname{belt-hash}}.
Используется алгоритм хэширования~\algname{belt-hash}, 
описанный в~\ref{PARAMS.Aux}.

{\bf Функция~$h$}.
Используется функция хэширования~$h$, 
которая действует из $\{0,1\}^{*}$ в~$\{0,1\}^{2l}$.
Требования к~$h$ определены в~\ref{COMMON.Hash}.

{\bf Одноразовый личный ключ~$k$}.
При выработке ЭЦП используется одноразовый личный 
ключ~$k\in\{1,2,\ldots,q-1\}$.
Требования по управлению~$k$ определены в~\ref{COMMON.Keys}.

{\bf Переменная $H$}.
Используется переменная~$H\in\{0,1\}^{2l}$.

{\bf Переменная $R$}.
Используется переменная~$R\in E_{a,b}(\FF_p)$.

{\bf Переменная $t$}.
При проверке ЭЦП используется переменная~$t\in\{0,1\}^l$.

\subsection{Алгоритм выработки электронной цифровой подписи}
\label{SIGN.Sign}

ЭЦП составляется из частей~$S_0\in\{0,1\}^{l}$ и~$S_1\in\{0,1\}^{2l}$. 
Выработка ЭЦП состоит в выполнении следующих шагов:
\begin{enumerate}
\item
Установить $H\leftarrow h(X)$.

\item
Выработать
$k\stackrel{R}{\leftarrow}\{1,2,\ldots,q-1\}$
(в соответствии с требованиями~\ref{COMMON.Keys}).

\item
Установить $R\leftarrow kG$.

\item
Установить
$S_0\leftarrow 
\bigl\langle
\algname{belt-hash}
(\OID(h)\parallel\langle R\rangle_{2l}\parallel H)
\bigr\rangle_{l}$.

\item
Установить
$S_1\leftarrow\bigl\langle(k-\overline{H}-(\overline{S}_0+2^{l})d)\bmod q
\bigr\rangle_{2l}$.

\item
Установить $S\leftarrow S_0\parallel S_1$.

\item
Возвратить $S$.
\end{enumerate}

\subsection{Алгоритм проверки электронной цифровой подписи}
\label{SIGN.Verify}

Проверка ЭЦП состоит в выполнении следующих шагов:
\begin{enumerate}
\item
Если $|S|\neq 3l$,
то возвратить \texttt{НЕТ}.

\item
Представить $S$ в виде $S=S_0\parallel S_1$,
где $S_0\in\{0,1\}^{l}$, $S_1\in\{0,1\}^{2l}$.

\item
Если $\overline{S}_1\geq q$,
то возвратить \texttt{НЕТ}.

\item
Установить $H\leftarrow h(X)$.

\item
Установить
$R\leftarrow \left((\overline{S}_1+\overline{H})\bmod q\right)G+
(\overline{S}_0+2^{l})Q$.

\item
Если $R=O$, то возвратить \texttt{НЕТ}.

\item
Установить 
$t\leftarrow\bigl\langle
\algname{belt-hash}
(\OID(h)\parallel\langle R\rangle_{2l}\parallel H)
\bigr\rangle_l$.

\item
Если $S_0\neq t$,
то возвратить \texttt{НЕТ}.

\item
Возвратить \texttt{ДА}.
\end{enumerate}
