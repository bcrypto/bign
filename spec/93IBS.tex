\begin{appendix}{В}{рекомендуемое}
{Идентификационная электронная цифровая подпись}
\label{IBS}

\hiddensection{Назначение}

Идентификационная ЭЦП (далее~--- ИЭЦП), 
предложенная в~\cite{IBS}, 
может использоваться в системах с централизованным 
управлением личными и открытыми ключами сторон.
%
В настоящем приложении определяются алгоритмы ИЭЦП,
основанные на вычислениях в группе точек эллиптической кривой.

Проверка ИЭЦП объединяет контроль целостности 
и подлинности сообщения с контролем подлинности подписавшей его стороны. 
Для сравнения, в инфраструктурах открытых ключей
на основе сертификатов СТБ~34.101.19 требуется проверять ЭЦП
по крайней мере дважды: сначала подпись сертификата, а затем 
непосредственно подпись сообщения.

Для организации ИЭЦП требуется наличие доверенной стороны. 
Доверенная сторона вырабатывает личный ключ~$d$ 
и соответствующий открытый ключ~$Q$ с помощью алгоритма~\ref{GENKEYPAIR}.
Остальные стороны получают свои ключи от доверенной стороны.

Стороны снабжаются уникальными идентификаторами.
Сторона с идентификатором~$Id$ обращается к доверенной стороне
с запросом на получение ключей. Доверенная сторона проверяет
подлинность~$Id$, подписывает~$Id$ на ключе~$d$ с помощью 
алгоритма~\ref{SIGN.Sign} и возвращает подпись обратно. 
%
Правила назначения идентификаторов и методы проверки их подлинности
определяются в конкретной информационной системе.

Подпись позволяет определить личный ключ обратившейся стороны 
и поэтому должна возвращаться конфиденциально.
Доверенная сторона должна контролировать подписываемые ею данные
и не должна раскрывать подписи сообщений, 
которые могут быть идентификаторами.

Сторона с идентификатором~$Id$ проверяет 
полученную подпись на ключе~$Q$ 
и одновременно извлекает из нее свои личный ключ~$e$ 
и открытый ключ~$R$, используя алгоритм~\ref{IBS.Extract}.
%
Доверенная сторона также может определить эти ключи, и
таким образом ИЭЦП сохраняет свойства обычной ЭЦП 
только тогда, когда доверенная сторона 
не использует~$e$ для подписи сообщений от~чужого имени.

После извлечения ключей можно подписывать сообщения. 
%
Подпись вырабатывается с помощью алгоритма~\ref{IBS.Sign},
а проверяется с помощью алгоритма~\ref{IBS.Verify}. 
%
При выработке подписи используются идентификатор~$Id$ 
и личный ключ~$e$. 
%
При проверке подписи используются открытые ключи~$Q$ и~$R$,
а также идентификатор~$Id$. 
%
Подпись передается и хранится вместе  
с открытым ключом~$R$ (см.~\ref{IBS.Format}).

Личный ключ~$e$ разрешено использовать только 
в алгоритме выработки ИЭЦП
(в том числе для генерации одноразового ключа, 
см.~\ref{IBS.Transforms}).

\hiddensection{Алгоритмы}\label{IBS.Algs}

\subsection{Входные и выходные данные}

Входными данными алгоритмов ИЭЦП являются 
параметры $p$, $a$, $b$, $q$, $G$, которые описывают группу 
точек эллиптической кривой.
Параметры должны удовлетворять условиям алгоритма~\ref{VALEC}. 
По модулю~$p$ определяется уровень стойкости~$\ell$ как минимальное 
натуральное число, для которого $p<2^{2\ell}$.

Кроме параметров эллиптической кривой,
входными данными алгоритма извлечения пары ключей
являются идентификатор~$Id\in\{0,1\}^*$,
подпись~$S\in\{0,1\}^*$ идентификатора, 
выработанная доверенной стороной, 
и открытый ключ~$Q\in E_{a,b}^*(\FF_p)$ доверенной стороны.
%
Открытый ключ~$Q$ должен удовлетворять условиям
алгоритма~\ref{VALPUBKEY}.

Выходными данными алгоритма извлечения пары ключей
является либо признак~\texttt{ОШИБКА}, либо пара ключей:
личный ключ~$e\in\{0,1,\ldots,q-1\}$ и 
открытый ключ~$R\in E_{a,b}^*(\FF_p)$.
%
Возврат признака ошибка означает, что~$S$ 
не является корректной подписью~$Id$ и ключи 
не могут быть из нее извлечены.

Кроме параметров эллиптической кривой,
входными данными алгоритма выработки ИЭЦП
являются 
идентификатор~$Id\in\{0,1\}^*$,
сообщение~$X\in\{0,1\}^*$
и личный ключ~$e\in\{0,1,\ldots,q-1\}$.

Выходными данными алгоритма выработки ИЭЦП
является слово~$S\in\{0,1\}^{3\ell}$~--- 
идентификационная подпись~$X$.

Кроме параметров эллиптической кривой,
входными данными алгоритма проверки ИЭЦП
являются идентификатор~$Id\in\{0,1\}^*$, 
сообщение~$X\in\{0,1\}^*$,
идентификационная подпись~$S\in\{0,1\}^*$, 
открытый ключ~$R=(x_R,y_R)$, где~$x_R$, $y_R$~--- целые числа, 
и открытый ключ~$Q\in E_{a,b}^*(\FF_p)$ доверенной стороны.
%
Открытый ключ~$Q$ должен удовлетворять условиям
алгоритма~\ref{VALPUBKEY}.

Выходными данными алгоритма проверки ИЭЦП
является ответ~\texttt{ДА} или~\texttt{НЕТ}.
Ответ~\texttt{ДА} означает, что~$S$ 
является корректной подписью~$X$,
выработанной стороной с идентификатором~$Id$.
Ответ~\texttt{НЕТ} означает обратное.

\subsection{Вспомогательные алгоритмы и преобразования, 
переменные}\label{IBS.Transforms}

{\bf Алгоритм~\algname{belt-hash}}.
Используется алгоритм хэширования~\algname{belt-hash}, 
описанный в~\ref{PARAMS.Aux}.

{\bf Проверка точки эллиптической кривой}.
На шаге~4 алгоритма проверки ИЭЦП контролируется,
что открытый ключ~$R$ является аффинной точкой эллиптической кривой.
Контроль может быть выполнен с помощью алгоритма
проверки открытого ключа, определенного в~\ref{VALPUBKEY}.

{\bf Функция~$h$}.
Используется функция хэширования~$h$, 
которая действует из $\{0,1\}^{*}$ в~$\{0,1\}^{2\ell}$.
Требования к~$h$ определены в~\ref{COMMON.Hash}.

{\bf Одноразовый личный ключ~$k$}.
При выработке ИЭЦП используется одноразовый личный 
ключ~$k\in\{1,2,\ldots,q-1\}$.
Требования по управлению~$k$ определены в~\ref{COMMON.Keys}.
Разрешается вырабатывать~$k$ с помощью алгоритма~\ref{GENK.Alg}
при замене~$d$ на~$e$ и с синхропосылкой~$H$, 
полученной на шаге~2 алгоритма~\ref{IBS.Sign}.

{\bf Переменные~$H_0, H$}.
Используются переменные~$H_0, H\in\{0,1\}^{2\ell}$.

{\bf Переменная~$V$}.
Используется переменная~$V\in E_{a,b}(\FF_p)$.

{\bf Переменная~$t$}.
При проверке ИЭЦП используется переменная~$t\in\{0,1\}^\ell$.

\subsection{Алгоритм извлечения пары ключей}
\label{IBS.Extract}

Извлечение пары ключей состоит в выполнении следующих шагов:
\begin{enumerate}
\item
Если $|S|\neq 3\ell$,
то возвратить \texttt{ОШИБКА}.

\item
Представить $S$ в виде $S=S_0\parallel S_1$,
где $S_0\in\{0,1\}^\ell$, $S_1\in\{0,1\}^{2\ell}$.

\item
Если~$\overline{S}_1\geq q$,
то возвратить~\texttt{ОШИБКА}.

\item
Установить $H_0\leftarrow h(Id)$.

\item
Установить $e\leftarrow(\overline{S}_1+\overline{H}_0)\bmod q$.

\item
Установить
$V\leftarrow eG + (\overline{S}_0+2^\ell)Q$.

\item
Если $V=O$, то возвратить \texttt{ОШИБКА}.

\item
Установить $t\leftarrow
\bigl\langle\algname{belt-hash}
(\OID(h)\parallel\langle V\rangle_{2\ell}\parallel H_0)
\bigr\rangle_\ell$.

\item
Если $S_0\neq t$,
то возвратить \texttt{ОШИБКА}.

\item
Установить $R\leftarrow V$.

\item
Возвратить~$(e,R)$.
\end{enumerate}

\subsection{Алгоритм выработки идентификационной 
электронной цифровой подписи}\label{IBS.Sign}

ИЭЦП составляется из 
частей~$S_0\in\{0,1\}^\ell$ и~$S_1\in\{0,1\}^{2\ell}$. 
Выработка ИЭЦП состоит в выполнении следующих шагов:
\begin{enumerate}
\item
Установить $H_0\leftarrow h(Id)$.

\item
Установить $H\leftarrow h(X)$.

\item
Выработать
$k\stackrel{R}{\leftarrow}\{1,2,\ldots,q-1\}$
(в соответствии с требованиями~\ref{COMMON.Keys}).

\item
Установить $V\leftarrow kG$.

\item
Установить
$S_0\leftarrow
\bigl\langle\algname{belt-hash}
(\OID(h)\parallel\langle V\rangle_{2\ell}\parallel H_0\parallel H)
\bigr\rangle_\ell$.

\item
Установить
$S_1\leftarrow\left\langle(k-\overline{H}-(\overline{S}_0+2^\ell)e)\bmod q
\right\rangle_{2\ell}$.

\item
Установить $S\leftarrow S_0\parallel S_1$.

\item
Возвратить $S$.
\end{enumerate}

\subsection{Алгоритм проверки идентификационной 
электронной цифровой подписи}
\label{IBS.Verify}

Проверка ИЭЦП состоит 
в выполнении следующих шагов:
\begin{enumerate}
\item
Если $|S|\neq 3\ell$,
то возвратить \texttt{НЕТ}.

\item
Представить $S$ в виде $S=S_0\parallel S_1$,
где $S_0\in\{0,1\}^\ell$, $S_1\in\{0,1\}^{2\ell}$.

\item
Если $\overline{S}_1\geq q$,
то возвратить \texttt{НЕТ}.

\item
Проверить, что $R\in E_{a,b}^*(\FF_p)$.
Если условие не выполняется, то возвратить~\texttt{НЕТ}.

\item
Установить $H_0\leftarrow h(Id)$.

\item
Установить $H\leftarrow h(X)$.

\item
Установить $t\leftarrow
\bigl\langle\algname{belt-hash}
(\OID(h)\parallel\langle R\rangle_{2\ell}\parallel H_0)
\bigr\rangle_\ell$.

\item
Установить
$V\leftarrow \left((\overline{S}_1+\overline{H})\bmod q\right)G+
(\overline{S}_0+2^\ell)(R-(\overline{t} + 2^\ell)Q)$.

\item
Если $V=O$, то возвратить \texttt{НЕТ}.
         
\item
Установить $t\leftarrow
\bigl\langle\algname{belt-hash}
(\OID(h)\parallel\langle V\rangle_{2\ell}\parallel H_0\parallel H)
\bigr\rangle_\ell$.

\item
Если $S_0\neq t$,
то возвратить \texttt{НЕТ}.

\item
Возвратить \texttt{ДА}.
\end{enumerate}
               

\hiddensection{Объединение идентификационной 
электронной цифровой подписи с открытым ключом}
\label{IBS.Format}

При передаче и хранении ИЭЦП~$S\in\{0,1\}^{3\ell}$
должна объединяться с открытым ключом~$R\in E_{a,b}^*(\FF_p)$ 
в виде слова~$S\parallel\langle R\rangle_{4\ell}$.

\end{appendix}
