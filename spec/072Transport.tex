\section{Транспорт ключа}\label{TRANSPORT}

\subsection{Входные и выходные данные}\label{TRANSPORT.InOut}
	
Входными данными алгоритмов транспорта ключа являются 
параметры $p$, $a$, $b$, $q$, $G$, которые описывают группу 
точек эллиптической кривой.
Параметры должны удовлетворять условиям алгоритма~\ref{VALEC}.
По модулю~$p$ определяется уровень стойкости~$\ell$ как минимальное 
натуральное число, для которого $p<2^{2\ell}$.

Кроме параметров эллиптической кривой,
входными данными алгоритма создания токена~ключа являются
транспортируемый ключ~$X\in\{0,1\}^{8*}$,
его заголовок~$I\in\{0,1\}^{128}$
и открытый ключ~$Q\in E_{a,b}^*(\FF_p)$ получателя~$X$.
%
Длина~$X$ должна быть не меньше $128$.
%
Открытый ключ~$Q$ должен удовлетворять условиям 
алгоритма~\ref{VALPUBKEY}.

Выходными данными алгоритма создания токена является 
слово $Y\in\{0,1\}^{2\ell+|X|+128}$~--- токен ключа~$X$.

Кроме параметров эллиптической кривой,
входными данными алгоритма разбора токена~ключа являются 
токен~$Y\in\{0,1\}^*$,
заголовок~$I\in\{0,1\}^{128}$ транспортируемого в нем ключа
и личный ключ~$d\in\{1,2,\ldots,q-1\}$ получателя токена.

Выходными данными алгоритма разбора токена 
является либо признак~\texttt{ОШИБКА},
либо слово~$X\in\{0,1\}^{|Y|-2\ell-128}$~---
ключ, который транспортируется в токене~$Y$.
%
Возврат признака~\texttt{ОШИБКА} означает некорректность токена.

\subsection{Вспомогательные алгоритмы и переменные}
\label{TRANSPORT.Aux}

{\bf Алгоритм~\algname{belt-keywrap}}.
Используется алгоритм~\algname{belt-keywrap},
определенный в СТБ~34.101.31 (пункт~6.8.3).
Входными данными алгоритма являются
транспортируемый ключ~$X\in\{0,1\}^{8*}$, заголовок $I\in\{0,1\}^{128}$
и ключ защиты~$\theta\in\{0,1\}^{256}$.
Выходными данными является
защищенный ключ~$Y\in\{0,1\}^{|X|+128}$.

{\bf Алгоритм~\algname{belt-keyunwrap}}.
Используется алгоритм~\algname{belt-keyunwrap},
определенный в СТБ~34.101.31 (пункт~6.8.4).
Входными данными алгоритма являются
защищенный ключ~$Y\in\{0,1\}^*$,
заголовок~$I\in\{0,1\}^{128}$ 
и ключ защиты~$\theta\in\{0,1\}^{256}$.
%
Выходными данными является
либо признак \texttt{ОШИБКА}, 
либо транспортируемый ключ~$X\in\{0,1\}^{|Y|-128}$.
%Возврат признака \texttt{ОШИБКА} означает нарушение
%целостности транспортируемого ключа.

{\bf Одноразовый личный ключ~$k$}.
При создании токена используется одноразовый личный 
ключ~$k\in\{1,2,\ldots,q-1\}$.
Требования по управлению~$k$ определены в~\ref{COMMON.Keys}.

{\bf Секретный ключ~$\theta$}.
Используется секретный ключ~$\theta\in\{0,1\}^{256}$.
Ключ~$\theta$ должен быть уничтожен после использования.

{\bf Переменная $R$}.
Используется переменная~$R=(x_R,y_R)\in E_{a,b}^*(\FF_p)$.

\subsection{Алгоритм создания токена ключа}\label{TRANSPORT.Wrap}

Алгоритм создания токена ключа состоит в выполнении следующих шагов:
\begin{enumerate}
\item
Выработать
$k\stackrel{R}{\leftarrow}\{1,2,\ldots,q-1\}$
(в соответствии с требованиями~\ref{COMMON.Keys}).

\item
Установить $R\leftarrow kG$.

\item
Установить $\theta\leftarrow\langle kQ\rangle_{256}$.

\item
Установить $Y\leftarrow \langle R\rangle_{2\ell}\parallel
\algname{belt-keywrap}(X,I,\theta)$.

\item
Возвратить $Y$.
\end{enumerate}

\subsection{Алгоритм разбора токена ключа}\label{TRANSPORT.Unwrap}

Алгоритм разбора токена ключа состоит в выполнении следующих шагов:
\begin{enumerate}
\item
Если длина $Y$ не кратна~$8$ или 
$|Y|<2\ell + 256$, то возвратить \texttt{ОШИБКА}.

\item
Представить $Y$ в виде $Y_0\parallel Y_1$,
где $Y_0\in\{0,1\}^{2\ell}$, 
$Y_1\in\{0,1\}^{|Y|-2\ell}$.

\item
Установить $x_R\leftarrow \overline{Y}_0$.

\item
Если $x_R\geq p$, то возвратить \texttt{ОШИБКА}.

\item
Установить $y_R\leftarrow (x_R^3+ax_R+b)^{(p+1)/4}\bmod p$.

\item
Если $y_R^2\not\equiv x_R^3+ax_R+b\pmod{p}$, то возвратить \texttt{ОШИБКА}.

\item
Построить $R=(x_R,y_R)$.

\item
Установить $\theta\leftarrow\langle dR\rangle_{256}$.

\item
Если $\algname{belt-keyunwrap}(Y_1,I,\theta)=\texttt{ОШИБКА}$, 
то возвратить \texttt{ОШИБКА}.

\item
Установить $X\leftarrow\algname{belt-keyunwrap}
(Y_1,I,\theta)$.

\item
Возвратить $X$.
\end{enumerate}

