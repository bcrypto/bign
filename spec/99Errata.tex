\clearpage
\chapter*{\mbox{}\hfill Поправка к официальной редакции\hfill\mbox{}}

\mbox{}

\begin{center}
\begin{tabular}{|p{3.3cm}|p{6.1cm}|p{6.1cm}|}
\hline
В каком месте & Напечатано & Должно быть\\
\hline
\hline
Раздел~\ref{TERMS},\par
термин <<целостность>>
&
Термин <<целостность>> определяется перед термином <<хэш-значение>>.
&
Термин <<целостность>> определяется после термина <<хэширование>>.
\\
\hline
Пункт~\ref{TRANSPORT.InOut},\par
абзац 3 
&
Входными данными алгоритма создания токена является 
слово $Y\in\{0,1\}^{2l+|X|+128}$~--- токен ключа~$X$. 
&
Выходными данными алгоритма создания токена является 
слово $Y\in\{0,1\}^{2l+|X|+128}$~--- токен ключа~$X$.
\\
\hline
Пункт~\ref{TRANSPORT.Wrap},\par
заголовок
&
Алгоритм генерации псевдослучайных чисел &
Алгоритм создания токена ключа
\\
\hline
Приложение~\ref{BER} &
$L=\langle 128+r\rangle_8\parallel
\langle b_r\rangle_8\parallel
\langle b_{r-1}\rangle_8\parallel\ldots\parallel
\langle b_0\rangle_8$
&
$L=\langle 129+r\rangle_8\parallel
\langle b_r\rangle_8\parallel
\langle b_{r-1}\rangle_8\parallel\ldots\parallel
\langle b_0\rangle_8$
\\
\hline
Приложение~\ref{TEST},\par
таблица~\ref{Table.TEST.GENK1},\par
последняя строка &
\texttt{AE443163~32A85C3B~9F6B31EE} \texttt{EADFF088~D30FE507~021AC86A}
$\texttt{3EC8E087~4ED33648}_{16}$ 
&
\texttt{7ADC8713~283EBFA5~47A2AD9C} \texttt{DFB245AE~0F7B968D~F0F91CB7}
$\texttt{85D1F932~A3583107}_{16}$\\
\hline
Приложение~\ref{ASN},\par
подраздел~\ref{ASN.OIDs},\par
строка~11 &
\texttt{Bign-genk} &
\texttt{bign-genk}\\
%
\hline
Приложение~\ref{PASSWORD},\par
подраздел~\ref{PASSWORD.Common},\par
абзац~4 &
Алгоритм~\ref{PASSWORD.PBKDF.Alg} имеет самостоятельное значение.
Вырабатываемый с его помощью ключ~$\theta$ может использоваться
для защиты произвольных личных и секретных ключей с помощью алгоритмов, не 
обязательно совпадающих с алгоритмами~\ref{PASSWORD.Protect}. 
&
Алгоритм~\ref{PASSWORD.PBKDF.Alg} имеет самостоятельное значение.
Вырабатываемый с его помощью ключ~$\theta$ может использоваться
для защиты произвольных личных и секретных ключей,
а также сопровождающих их служебных данных.
Защита может выполняться с помощью алгоритмов, не обязательно совпадающих 
с алгоритмами~\ref{PASSWORD.Protect}.\\ 
%
\hline
Приложение~\ref{PASSWORD},\par
пункт~\ref{PASSWORD.PBKDF.Aux},\par
заголовок
&
Вспомогательные алгоритмы &
Вспомогательные алгоритмы и переменные\\
\hline
Приложение~\ref{PASSWORD},\par
пункт~\ref{PASSWORD.PBKDF.Aux},\par
дополнительный абзац
&
&
Используется переменная~$t\in\{0,1\}^{256}$. 
Значение~$t$ должно быть уничтожено после использования.\\
\hline
\end{tabular}

\vfill

\mbox{}

\begin{tabular}{|p{3.3cm}|p{6.1cm}|p{6.1cm}|}
\hline
В каком месте & Напечатано & Должно быть\\
\hline
\hline
Приложение~\ref{PASSWORD},\par
пункт~\ref{PASSWORD.PBKDF.Alg},\par
2-й и последующий шаги алгоритма &
%
2~Для $i=1,2,\ldots,c$ выполнить:\par
\mbox{}~~1)~$\theta\leftarrow\HMAC_{\algname{belt-hash}}(P,\theta)$.\par
3~Возвратить~$\theta$.
&
2~Установить $t\leftarrow\theta$.\par
3~Для $i=2,3,\ldots,c$ выполнить:\par
\mbox{}~~1)~$t\leftarrow\HMAC_{\algname{belt-hash}}(P,t)$;\par
\mbox{}~~2)~$\theta\leftarrow\theta\oplus t$.\par
4~Возвратить~$\theta$.\\
%
\hline
Приложение~\ref{PASSWORD},\par
таблица~\ref{Table.PASSWORD.TestPBKDF},\par
строка~3 &
%
\texttt{BE329713~43FC9A48~A02A885F} $\texttt{194B09A1}_{16}$ &
%
$\texttt{BE329713~43FC9A48}_{16}$\\
\hline
Приложение~\ref{PASSWORD},\par
таблица~\ref{Table.PASSWORD.TestPBKDF},\par
строка~4 &
%
\texttt{D9024724~82130F3B~77D09303} \texttt{03DD7E4E~68630CC0~2B56A8B2}
$\texttt{AFA74F09~6BCAC971}_{16}$
&
\texttt{3D331BBB~B1FBBB40~E4BF22F6} \texttt{CB9A689E~F13A77DC~09ECF932}
$\texttt{91BFE424~39A72E7D}_{16}$
\\
\hline
Приложение~\ref{PASSWORD},\par
таблица~\ref{Table.PASSWORD.TestPBKDF},\par
строки 5, 6 &
%
\texttt{248E0CD7~639B1237~76F1CEC1} \texttt{FCECE708~C2DFC53F~78ECEA6C}
\texttt{33B4C3C1~E6183AD6~D8A18CFA} $\texttt{F540976E~1022B89D~BA32DA18}_{16}$
&
\texttt{4EA289D5~F718087D~D8EDB305} \texttt{BA1CE898~0E5EC3E0~B56C8BF9}
\texttt{D5C3E909~CF4C14F0~7B8204E6} $\texttt{7841A165~E924945C~D07F37E7}_{16}$
\\
\hline
Приложение~\ref{PASSWORD},\par
пункт~\ref{PASSWORD.Ids},\par
последний абзац &
Вложенный в \texttt{encryptionScheme} компонент \texttt{algorithm}
должен принимать значение \texttt{belt-keywrap256},
а компонент \texttt{parameters}~--- значение \texttt{NULL}.
&
Вложенный в \texttt{encryptionScheme} компонент \texttt{algorithm}
должен принимать значение \texttt{belt-keywrap256},
а компонент \texttt{parameters}~--- значение \texttt{NULL}.
%
Идентификатор \texttt{belt-keywrap256} 
определен в СТБ~34.101.31 (приложение~Б).
\\
\hline
\end{tabular}
\end{center}