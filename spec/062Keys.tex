\section{Генерация и проверка ключей}\label{KEYS}

\subsection{Входные и выходные данные}

Входными данными алгоритма генерации пары ключей
являются параметры~$p$, $a$, $b$, $q$, $G$, 
которые описывают группу точек эллиптической кривой.
Параметры должны удовлетворять условиям 
алгоритма~\ref{VALEC}.

Выходными данными алгоритма генерации пары ключей 
являются личный ключ $d\in\{1,2,\ldots,q-1\}$
и соответствующий открытый ключ $Q\in E_{a,b}^*(\FF_p)$.

Входными данными алгоритма проверки открытого ключа 
являются параметры~$p$, $a$, $b$, 
которые описывают группу точек эллиптической кривой,
и открытый ключ~$Q=(x_Q,y_Q)$, где $x_Q, y_Q$~--- целые числа.
%
Параметры эллиптической кривой должны удовлетворять условиям
алгоритма~\ref{VALEC}.

Выходными данными алгоритма проверки открытого ключа 
является ответ~\texttt{ДА} или \texttt{НЕТ}. 
Ответ~\texttt{ДА} означает, что~$Q$ является допустимым открытым ключом.
Ответ~\texttt{НЕТ} означает обратное.

\subsection{Алгоритм генерации пары ключей}\label{GENKEYPAIR}

Генерация пары ключей состоит в выполнении следующих шагов:
\begin{enumerate}
\item
Выработать $d\stackrel{R}{\leftarrow}\{1,2,\ldots,q-1\}$
(в соответствии с требованиями~\ref{COMMON.Keys}).

\item
Установить $Q\leftarrow dG$.

\item
Возвратить $(d,Q)$.
\end{enumerate}

\subsection{Алгоритм проверки открытого ключа}\label{VALPUBKEY}

Проверка открытого ключа состоит в выполнении следующих шагов:
\begin{enumerate}
\item
Если нарушается одно из условий:
\begin{enumerate}
\item
$0\leq x_Q,y_Q<p$;
\item
$y_Q^2\equiv x_Q^3+ax_Q+b\pmod{p}$,
\end{enumerate}
то возвратить \texttt{НЕТ}.

\item
Возвратить \texttt{ДА}.
\end{enumerate}

