\chapter{Обозначения}\label{DEFS}

\section{Список обозначений}

{\tabcolsep 0pt
\begin{longtable}{lrp{13.5cm}}
$\{0,1\}^n$  & \hspace{2mm} &
множество всех слов длины $n$ в алфавите~$\{0,1\}$;
\\[4pt]
$\{0,1\}^*$  &&
множество всех слов конечной длины в алфавите~$\{0,1\}$
(включая пустое слово длины $0$);
\\[4pt]
$|u|$      &&
длина слова $u\in\{0,1\}^*$;
\\[4pt]
%
$\{0,1\}^{n*}$  &&
множество всех слов из~$\{0,1\}^*$,
длина которых кратна~$n$;
\\[4pt]
%
$\alpha^n$  &&
для~$\alpha\in\{0,1\}$
слово длины~$n$ из одинаковых символов~$\alpha$;
\\[4pt]
%
%
$\left\langle u\right\rangle_n$  &&
для $u\in\{0,1\}^*$
слово из первых~$n$ символов~$u$, $n\leq|u|$;
\\[4pt]
%
$u\parallel v$  &&
конкатенация
$u_1 u_2\ldots u_n v_1 v_2\ldots v_m$
слов
$u=u_1 u_2\ldots u_n$ и
$v=v_1 v_2\ldots v_m$;
\\[4pt]
%
$\texttt{01234\ldots}_{16}$ && 
представление $u\in\{0,1\}^{4*}$ шестнадцатеричным словом,
при котором последовательным четырем символам~$u$ соответствует
один шестнадцатеричный символ
(например, $10100010=\texttt{A2}_{16}$);
\\[4pt]
%
$x\bmod m$             &&
для целого числа~$x$ и натурального числа~$m$ 
остаток от деления $x$ на $m$,
т.~е. число $r\in\{0,1,\ldots,m-1\}$ такое, что $m$ делит $x-r$;
\\[4pt]
%
$x\equiv y\pmod m$    &&
$x$ сравнимо с $y$ по модулю $m$, т.~е. $x\bmod m=y\bmod m$;
\\[4pt]
%
$u\oplus v$             &&
для~$u=u_1 u_2\ldots u_n\in\{0,1\}^n$ 
и~$v=v_1 v_2\ldots v_n\in\{0,1\}^n$
слово~$w=w_1 w_2\ldots w_n\in\{0,1\}^n$
из символов~$w_i=(u_i+v_i)\bmod{2}$;
\\[4pt]
%
$\bar u$                &&
а)~для~$u=u_1 u_2\ldots u_8\in\{0,1\}^8$
число $2^7 u_1+2^6 u_2+\ldots+u_8$ и\\[2pt]
%
                        &&
б)~для~$u=u_1\parallel u_2\parallel\ldots\parallel u_n$, $u_i\in\{0,1\}^8$,
число~$\bar u_1+2^8\bar u_2+\ldots+2^{8(n-1)}\bar u_n$;
\\[4pt]
%
$\langle U\rangle_{8n}$ &&
для целого числа~$U$ 
слово $u\in\{0,1\}^{8n}$ такое, что $\bar u=U\bmod 2^{8n}$;
\\[4pt]
%
$u\boxplus v$           &&
для~$u,v\in\{0,1\}^{8n}$ слово $\langle\bar u+\bar v\rangle_{8n}$;
\\[4pt]
%
$\FF_p$               &&
для простого числа~$p$ множество $\{0,1,\ldots,p-1\}$
с операциями сложения и умножения по модулю~$p$,
конечное поле из $p$ элементов;
\\[4pt]
%
$\left(\frac{u}{p}\right)$   &&
для нечетного простого числа~$p$ и~$u\in\FF_p$ 
символ Лежандра:
$0$,~если $u=0$;
$1$,~если $u$~--- квадратичный вычет по модулю~$p$,
и~$-1$ в остальных случаях;
\\[4pt]
%
$E_{a,b}^*(\FF_p)$       &&
для~$a,b\in\FF_p$ множество решений~$(x,y)$, $x,y\in\FF_p$, 
уравнения~$y^2=x^3+ax+b$,
множество аффинных точек эллиптической кривой;
\\[4pt]
%
$O$       &&
бесконечно удаленная точка;
\\[4pt]
%
$E_{a,b}(\FF_p)$       &&
множество $E_{a,b}^*(\FF_p)\cup\{O\}$ с операцией сложения точек,
группа точек эллиптической кривой;
\\[4pt]
%
$kP$       &&
для~$P\in E_{a,b}(\FF_p)$ сумма~$k$ экземпляров~$P$,
кратная~$P$ точка;
\\[4pt]
%
$l$       &&
уровень стойкости,
число из множества~$\{128, 192, 256\}$;
\\[4pt]
%
$\langle P\rangle$  &&
для~$P=(x,y)\in E_{a,b}^*(\FF_p)$, 
где $2^{2l-1}<p<2^{2l}$, 
слово~$\langle x\rangle_{2l}\parallel 
\langle y\rangle_{2l}$;
\\[4pt]
%
$\langle P\rangle_{n}$    &&
для~$P\in E_{a,b}^*(\FF_p)$
слово из первых~$n$ символов~$\langle P\rangle$,
$n\leq \left|\langle P\rangle\right|$;
\\[4pt]
%
$c\leftarrow u$         &&
присвоение переменной $c$ значения $u$;
\\[4pt]
%
$c\stackrel{R}{\leftarrow} U$    &&
случайный равновероятный (или псевдослучайный)
выбор~$c$ из множества~$U$;
\\[4pt]
%
\algname{belt-hash} &&
алгоритм хэширования, определенный в СТБ~34.101.31 
(пункт 6.9.3);
\\[4pt]
%
$\OID(D)$    &&
кодовое представление идентификатора объекта~$D$,
полученное в соответствии с ГОСТ~34.973, ГОСТ~34.974.
\\[4pt]
\end{longtable}
} % tabcolsep
\setcounter{table}{0}

\section{Пояснения к обозначениям}

\subsection{Слова}

Двоичные слова представляют собой последовательности символов из 
алфавита~$\{0,1\}$. Символы нумеруются слева направо от единицы.
%
В настоящем подразделе в качестве примера рассматривается слово
$$
w=1011 0001 1001 0100 1011 1010 1100 1000.
$$
В этом слове первый символ~--- $1$, 
второй~--- $0$, \ldots, последний~--- $0$.

Слова разбиваются на тетрады из четверок последовательных двоичных символов.
%
Тетрады кодируются шестнадцатеричными символами по следующим правилам
(см. таблицу~\ref{Table.Hex}):

\begin{table}[h]
\caption{}\label{Table.Hex}
\begin{tabular}{|c|c||c|c||c|c||c|c|}
\hline
Тетрада & Символ & Тетрада & Символ & Тетрада & Символ & Тетрада & Символ\\
\hline
\hline
0000 & $\texttt{0}_{16}$ & 0001 & $\texttt{1}_{16}$ & 
0010 & $\texttt{2}_{16}$ & 0011 & $\texttt{3}_{16}$\\
0100 & $\texttt{4}_{16}$ & 0101 & $\texttt{5}_{16}$ & 
0110 & $\texttt{6}_{16}$ & 0111 & $\texttt{7}_{16}$\\ 
1000 & $\texttt{8}_{16}$ & 1001 & $\texttt{9}_{16}$ & 
1010 & $\texttt{A}_{16}$ & 1011 & $\texttt{B}_{16}$\\ 
1100 & $\texttt{C}_{16}$ & 1101 & $\texttt{D}_{16}$ & 
1110 & $\texttt{E}_{16}$ & 1111 & $\texttt{F}_{16}$\\ 
\hline
\end{tabular}
\end{table}

Пары последовательных тетрад образуют октеты.
Последовательные октеты слова~$w$ имеют вид:
$$
1011 0001=\texttt{B1}_{16},\ 
1001 0100=\texttt{94}_{16},\ 
1011 1010=\texttt{BA}_{16},\  
1100 1000=\texttt{C8}_{16}.
$$

\subsection{Слова как числа}

Октету $u=u_1 u_2\ldots u_8$ ставится в соответствие байт~--- 
число $\bar{u}=2^7u_1+2^6 u_2+\ldots + u_8$. 
Например, октетам $w$ соответствуют байты
$$
177=2^7+2^5+2^4+1,\ 
148=2^7+2^4+2^2,\ 
186=2^7+2^5+2^4+2^3+2^1,\ 
200=2^7+2^6+2^3.
$$

Число ставится в соответствие не только октетам, но и любому другому
двоичному слову, длина которого кратна~$8$. 
%
При этом используется распространенное для многих современных 
процессоров соглашение <<от младших к старшим>> (little-endian):
считается, что первый байт является младшим, последний~--- старшим.
Например, слову $w$ соответствует число
$$
\bar{w}=177+2^{8}\cdot 148+2^{16}\cdot 186+2^{24}\cdot 200 = 3367670961.
$$

\subsection{Конечные поля}

Элементы~$\FF_p$ складываются и умножаются 
как целые числа с заменой результата на остаток от его деления на~$p$.
Множество~$\FF_p$ с такими операциями является конечным простым полем.
Нулевым элементом поля является число~$0$,
а мультипликативной единицей~--- число~$1$
(подробнее см.~\cite{LidNid88}).

Кроме сложения и умножения, в поле~$\FF_p$ можно выполнять вычитание и деление. 
%
Вычитание $u$ состоит в сложении с $p-u$.
%
Деление на $u\in\{1,2,\ldots,p-1\}$ состоит в 
умножении на число~$v\in\{1,2,\ldots,p-1\}$ такое, 
что $uv\equiv 1\pmod{p}$.

Например, в поле~$\FF_7$ выполняется:
$$
4+5=2,\quad
4\cdot 5=6,\quad
4-5=4+(7-5)=6,\quad
4/5 = 4\cdot 3=5.
$$

Квадраты ненулевых элементов~$\FF_p$ 
называются квадратичными вычетами по модулю~$p$. 
Например, имеется~$3$ квадратичных вычета по модулю~$7$:
$$
1=1^2,\quad
2=3^2,\quad
4=2^2.
$$
Поэтому 
$\left(\frac{1}{7}\right)=\left(\frac{2}{7}\right)=\left(\frac{4}{7}\right)=1$.
Кроме того, $\left(\frac{0}{7}\right)=0$ и 
$\left(\frac{3}{7}\right)=\left(\frac{5}{7}\right)=\left(\frac{6}{7}\right)=-1$.

\subsection{Эллиптические кривые}

Пусть~$p>3$ и~$4a^3+27b^2\not\equiv 0\pmod{p}$.
Множество~$E^*_{a,b}(\FF_p)$ состоит из решений уравнения 
$y^2=x^3+ax+b$ относительно~$x,y\in\FF_p$. 
Уравнение такого вида определяет эллиптическую кривую над полем~$\FF_p$,
его решения~$(x,y)$ называются аффинными точками кривой.
К аффинным точкам добавляется специальная бесконечно удаленная
точка~$O$ и образуется множество~$E_{a,b}(\FF_p)$
(подробнее см.~\cite{ECC}).
%
Например,
$$
E_{4,1}(\FF_{7})=\{O,(0,1),(0,6),(4,2),(4,5)\}.
$$

Множество~$E_{a,b}(\FF_p)$ является аддитивной группой 
при следующих правилах сложения:
\begin{enumerate}
\item
$O+P=P+O=P$ для всех $P\in E_{a,b}(\FF_p)$.

\item
Если $P=(x,y)\in E_{a,b}^*(\FF_p)$, то $-P=(x,p-y)$ и $P+(-P)=O$.

\item
Если $P_1=(x_1,y_1)\in E_{a,b}^*(\FF_p)$,
$P_2=(x_2,y_2)\in E_{a,b}^*(\FF_p)$ и~$P_2\neq -P_1$, 
то $P_1+P_2=(x_3,y_3)$,
где
$
x_3=\lambda^2-x_1-x_2,\quad
y_3=\lambda(x_1-x_3)-y_1,\quad
\lambda=\left\{
\begin{array}{rl}
\dfrac{y_2-y_1}{x_2-x_1}, & P_1\neq P_2,\\[12pt]
\dfrac{3x_1^2+a}{2 y_1}, & P_1=P_2
\end{array}
\right.
$\\
(вычисления ведутся в~$\FF_p$).
\end{enumerate}

Сумма $k$ экземпляров точки~$P$ называется $k$-кратной ей точкой 
и обозначается через~$kP$.
Например, для $P=(4,2)\in E_{4,1}(\FF_7)$ ее кратные имеют вид:
\begin{align*}
2P&=(4,2)+(4,2)=(0,1),\quad
&3P&=(0,1)+(4,2)=(0,6),\\
4P&=2(0,1)=(4,5),\quad
&5P&=(0,1)+(0,6)=O.
\end{align*}

Считается, что $0P=O$.

\subsection{Идентификаторы объектов}

В ГОСТ 34.973 определены правила абстрактно-синтаксической нотации 
версии~1 для описания различных информационных объектов. 
Эти правила регламентируют в том числе
присвоение объектам уникальных идентификаторов.

Идентификатор объекта представляет собой последовательность 
целых чисел. При записи идентификатора числа разделяются пробелами.
Вся последовательность окай\-мляется фигурными скобками.
Например, идентификатор алгоритма хэширования~\algname{belt-hash}
определен в СТБ~34.101.31 как~\{1~2~112~0~2~0~34~101~31~81\}.

Идентификатор объекта кодируется двоичным словом по правилам,
заданным в ГОСТ~34.974 и кратко изложенным в приложении~\ref{BER}.
Например, $\OID(\algname{belt-hash})=\texttt{06092A7000020022651F51}_{16}$.


