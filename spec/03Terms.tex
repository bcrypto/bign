\chapter{Термины и определения}\label{TERMS}

В настоящем стандарте применяют
следующие термины с соответствующими определениями:

{\bf \thedefctr~ключ}:
Параметр, который управляет криптографическими операциями 
выработки и проверки электронной цифровой подписи, 
зашифрования и расшифрования,
генерации псевдослучайных чисел и др.

{\bf \thedefctr~конфиденциальность}:
Гарантия того, что сообщения доступны для использования
только тем сторонам, которым они предназначены.
%
% ISO 1799: ensuring that information is accessible 
% only to those authorized to have access

{\bf \thedefctr~личный ключ:}
Ключ, который связан с конкретной стороной, не является общедоступным
и используется в настоящем стандарте для выработки электронной цифровой
подписи и для разбора токена ключа.

{\bf \thedefctr~октет}:
Двоичное слово длины~$8$.

{\bf \thedefctr~открытый ключ:}
Ключ, который строится по личному ключу, 
связан с конкретной стороной, может быть сделан общедоступным
и используется в настоящем стандарте для проверки электронной цифровой 
подписи и для создания токена ключа.

{\bf \thedefctr~подлинность}:
Гарантия того, что сторона действительно является
владельцем (создателем, отправителем) определенного сообщения.

{\bf \thedefctr~секретный ключ:}
Ключ, который связан с конкретными сторонами, не является общедоступным
и используется в настоящем стандарте для генерации псевдослучайных
чисел и для защиты других ключей.

{\bf \thedefctr~синхропосылка}:
Открытые входные данные криптографического алгоритма,
которые обеспечивают уникальность результатов 
криптографического преобразования на фиксированном ключе.

{\bf \thedefctr~сообщение}:
Двоичное слово конечной длины.

{\bf \thedefctr~токен ключа}:
Сообщение, которое передается от одной стороны другой
при транспорте ключа и представляет 
собой транспортируемый ключ в защищенной форме, а также данные,
необходимые получателю для снятия защиты.

{\bf \thedefctr~транспорт ключа}:
Конфиденциальная передача ключа от одной стороны другой.

{\bf \thedefctr~хэш-значение}:
Двоичное слово фиксированной длины, 
которое определяется по сообщению без использования ключа и 
служит для контроля целостности сообщения и для представления 
сообщения в (необратимо) сжатой форме.

{\bf \thedefctr~хэширование}:
Выработка хэш-значений.

{\bf \thedefctr~целостность}:
Гарантия того, что сообщение не изменено
при хранении или передаче.

{\bf \thedefctr~электронная цифровая подпись; ЭЦП}:
Контрольная характеристика сообщения, которая 
вырабатывается с использованием личного ключа,
проверяется с использованием открытого ключа,
служит для контроля целостности и подлинности сообщения
и обеспечивает невозможность отказа от авторства.
% Закон об ЭЦП и ЭД:
% электронная цифровая подпись – последовательность символов, являющаяся 
% реквизитом электронного документа и предназначенная для подтверждения его 
% целостности и подлинности


