\chapter{Общие положения}\label{COMMON}

\section{Назначение}\label{COMMON.Dest}

Настоящий стандарт определяет алгоритмы ЭЦП,
которые предназначены для контроля целостности и подлинности сообщений. 
%
Автор сообщения использует свой личный ключ для выработки ЭЦП, 
а связанный с личным ключом открытый ключ используется 
другими сторонами для проверки ЭЦП.
%
При правильном управлении ключами
корректность проверяемой подписи означает, 
что она была выработана владельцем личного ключа и после этого
сообщение не изменялось.
%
Только владелец личного ключа может выработать корректную 
ЭЦП, что не позволяет ему отказаться от авторства сообщения 
и может быть использовано другими сторонами для 
доказательства такого авторства.

\begin{note*}
Алгоритмы ЭЦП установлены также в СТБ~1176.2. Переход от алгоритмов СТБ~1176.2 к
алгоритмам стандарта позволит уменьшить время выработки и проверки ЭЦП,
сократить длины параметров и ключей при сохранении уровня криптографической
стойкости.
\end{note*}

Алгоритмы выработки и проверки ЭЦП 
построены по схеме Шнорра~\cite{Schnorr}.
%
При выполнении алгоритмов используются 
вычисления в группе точек эллиптической кривой
над конечным простым полем.
%
В стандарте определяются алгоритмы генерации 
и проверки параметров, описывающих искомую группу.
%
Определены также алгоритм генерации пары ключей (личного и открытого) 
и алгоритм проверки открытого ключа.

Алгоритмы проверки параметров эллиптической кривой и открытого
ключа следует применять в тех случаях, 
когда отсутствует гарантия их математической корректности.
%
%Например удостоверящий центр СТБ~34.101.19 при выдаче сертификата 
%открытого ключа должен проверить указанные в запросе на выдачу сертификата
%нестандартные долговременные параметры и открытый ключ.
%
Такая гарантия обеспечивает достоверность выводов о стойкости алгоритмов ЭЦП.
%
Вместе с тем
алгоритм проверки открытого ключа не гарантирует, 
что ключ действительно принадлежит определенной стороне
или что сторона знает соответствующий личный ключ.
%
Проверка знания личного ключа, 
удостоверение принадлежности открытого ключа 
и проверка такой принадлежности
реализуются с помощью дополнительных методов и средств, 
в совокупности называемых инфраструктурой открытых ключей.
%
Например, в СТБ~34.101.19 определяются элементы инфраструктуры 
на основе сертификатов открытых ключей.

Параметры эллиптической кривой, личный и открытый ключи могут 
быть использованы не только для контроля целостности и подлинности, 
но и для обеспечения конфиденциальности защищаемой информации.
%
В стандарте определяются алгоритмы транспорта ключа, 
предназначенные для защищенной передачи ключей 
и других секретных данных между двумя сторонами. 
%
С помощью транспортируемого ключа стороны могут выполнять 
шифрование или другие криптографические операции.

Для реализации транспорта отправитель 
выполняет алгоритм создания токена 
ключа и передает полученный токен получателю.
%Токен представляет собой сообщение, которое включает
%транспортируемый ключ в защищенной форме, а также данные,
%необходимые получателю для снятия защиты.
%
Получатель выполняет алгоритм разбора токена и восстанавливает
транспортируемый ключ.
%
При создании токена отправитель использует открытый ключ получателя.
При разборе токена получатель использует свой личный ключ.

В приложении~\ref{STD} приводятся стандартные наборы параметров
эллиптической кривой, которые были получены с помощью 
соответствующего алгоритма генерации 
и могут быть использованы напрямую, без повторного построения.

В приложении~\ref{IBS} определяются алгоритмы 
идентификационной~ЭЦП, с помощью которых в некоторых случаях можно 
упростить управление открытыми ключами.

В приложении~\ref{TEST} приводятся примеры выполнения 
алгоритмов стандарта.
Примеры можно использовать для проверки корректности реализаций 
алгоритмов.

В приложении~\ref{ASN} приводится модуль
абстрактно-синтаксической нотации версии~1 (АСН.1),
определенной в~ГОСТ 34.973.
Модуль задает идентификаторы алгоритмов и других объектов 
стандарта, описывает структуры данных для хранения 
ключей и параметров.
%
Рекомендуется использовать модуль 
при встраивании алгоритмов стандарта в информационные системы, 
в которых также используется АСН.1.
В частности, модуль может быть использован для уточнения 
форматов запроса на получение сертификата (определен в СТБ~34.101.17),
сертификатов и списков отозванных сертификатов (СТБ~34.101.19),
криптографических сообщений (СТБ~34.101.23),
запроса и ответа о статусе сертификата (СТБ~34.101.26).

В приложении~\ref{PASSWORD} определяются алгоритмы защиты личного ключа
на пароле его владельца. Алгоритмы соответствуют~\cite{PKCS5}.

В приложении~\ref{NT} определяются вспомогательные теоретико-числовые 
алгоритмы.

\section{Уровень стойкости}\label{COMMON.Strength}

Алгоритмы ЭЦП построены так, что злоумышленнику 
вычислительно трудно решить задачу подделки ЭЦП.
%
В этой задаче злоумышленник получает параметры эллиптической кривой 
и открытый ключ ЭЦП. Злоумышленник не знает личный ключ, 
но может передавать для подписи на нем произвольные сообщения, 
получать и анализировать результаты.
%
Ему требуется построить корректную ЭЦП к любому сообщению,
отличному от ранее подписанных.

Стойкость алгоритмов ЭЦП определяется уровнем~$\ell\in\{128,192,256\}$.
%
На уровне~$\ell$ для подделки ЭЦП злоумышленнику требуется выполнить 
порядка $2^\ell$ операций. 
%
Cтойкость основывается на сложности дискретного логарифмирования 
в группе точек эллиптической кривой и на стойкости используемых 
функций хэширования.

Уровень~$\ell$ определяет длины параметров, 
ключей, подписей, а также быстродействие алгоритмов ЭЦП. 
%
Следует учитывать, что с ростом~$\ell$, кроме повышения стойкости,
снижается быстродействие алгоритмов.

Для алгоритмов транспорта ключа вводятся аналогичные уровни 
стойкости~$\ell\in\{128,192,256\}$.
На уровне~$\ell$ для определения транспортируемого ключа 
по токену и открытому ключу получателя 
злоумышленнику требуется выполнить порядка~$2^\ell$ операций. 

\section{Параметры эллиптической кривой}\label{COMMON.Params}

{\bf Модуль~$p$.} 
Используется простое число $p$, которое удовлетворяет условиям:
$2^{2\ell-1}<p<2^{2\ell}$, $p\equiv 3\pmod{4}$.
%
Модуль определяет поле~$\FF_p$, 
над которым строится эллиптическая кривая.
%
Можно использовать произвольное допустимое~$p$, 
в том числе простое специального вида 
(например, $p=2^{2\ell}-c$, где $c$~--- малое натуральное число).

{\bf Коэффициенты $a$, $b$.} 
Используются числа $a,b\in\FF_p$, которые удовлетворяют условиям:
$a\neq 0$,
$\left(\frac{b}{p}\right)=1$,
$4a^3+27b^2\not\equiv 0\pmod{p}$.
%
Коэффициенты~$a$, $b$ вместе с модулем~$p$ 
определяют группу точек эллиптической кривой~$E_{a,b}(\FF_p)$.

{\bf Параметр~$seed$.} 
Числа~$p$ и~$a$ выбираются, а~$b$ строится по ним.
При построении~$b$ используется дополнительный
параметр~$seed\in\{0,1\}^{64}$,
который может быть выбран произвольным образом.

{\bf Порядок~$q$.}
После построения группы~$E_{a,b}(\FF_p)$ рассчитывается 
ее порядок~$q=|E_{a,b}(\FF_p)|$.
%
Выбирается группа, порядок которой удовлетворяет следующим ограничениям:
$q$~--- простое,
$2^{2\ell-1}<q<2^{2\ell}$,
$q\neq p$,
$q$ не делит числа вида $p^m-1$ для~$m=1,2,\ldots,50$.

{\bf Базовая точка $G$.}
Используется базовая
точка $G\in E_{a,b}^*(\FF_p)$ вида $G=(0,y_G)$,
где~$y_G=b^{(p+1)/4}\bmod{p}$.
%
Кратные $G$, $2G$,\ldots, $(q-1)G$ базовой точки
пробегают все элементы~$E_{a,b}^*(\FF_p)$, а $qG=O$.

Алгоритм генерации параметров эллиптической кривой определен в~\ref{GENEC}. 
Алгоритм проверки параметров определен в~\ref{VALEC}. 

\begin{note*}
Алгоритм генерации параметров имеет высокую вычислительную сложность. Основные
издержки связаны с расчетом порядка~$q$.
%
Алгоритм проверки параметров имеет значительно меньшую сложность, поскольку
требуется проверять, а не определять~$q$.
\end{note*}

\section{Ключи}\label{COMMON.Keys}

Личным ключом является число~$d\in\{1,2,\ldots,q-1\}$.
По личному ключу определяется открытый ключ~$Q\in E_{a,b}^*(\FF_p)$.
Алгоритм генерации личного и открытого ключей определен в~\ref{GENKEYPAIR}.
Алгоритм проверки открытого ключа определен в~\ref{VALPUBKEY}.

При хранении и распространении
должны обеспечиваться конфиденциальность и контроль целостности личного 
ключа, контроль целостности открытого ключа.
%
Применяемые методы управления ключами должны 
гарантировать принадлежность открытого ключа стороне, 
подпись которой проверяется, и владение данной стороной 
соответствующим личным ключом.

Кроме личного ключа~$d$, в алгоритме выработки ЭЦП 
используется одноразовый личный ключ~$k\in\{1,2,\ldots,q-1\}$.
Одноразовый ключ используется также в алгоритме создания токена ключа.
Одноразовые личные ключи должны вырабатываться без возможности 
предсказания и уничтожаться после использования.

Для создания личных ключей может быть использован 
физический генератор случайных чисел, удовлетворяющий ТНПА,
или алгоритм генерации псевдослучайных чисел,
определенный в СТБ~34.101.47 или в другом ТНПА. 
Входные данные алгоритма должны включать секретный ключ,
доступный только владельцу личного ключа, и уникальную 
синхропосылку. Длина ключа алгоритма генерации должна быть не меньше~$\ell$.

\begin{note*}
Личные ключи~--- числа из множества~$\{1,2,\ldots,q-1\}$~--- 
генерируются, как правило, в два этапа: cначала строятся случайные 
или псевдослучайные двоичные слова, которые затем преобразуются в числа. 
Для генерации личных ключей по такой схеме 
рекомендуется строить слова~$u\in\{0,1\}^{2\ell}$ до тех пор, 
пока не будет выполнено условие~$\bar{u}\in\{1,2,\ldots,q-1\}$,
и объявлять окончательное число~$\bar{u}$ результатом генерации.
В среднем потребуется проверить $2^{2\ell}/(q-1)$ чисел-кандидатов.
\end{note*}

%Для генерации личных ключей по такой схеме 
%рекомендуется использовать алгоритм~\ref{RANDMOD.Alg}.
%                                     
%Если генерируемые случайные или псевдослучайные числа 
%представляют собой двоичные символы,
%то для их объединения и преобразования 
%в ключи из множества~$\{1,2,\ldots,q-1\}$ рекомендуется 
%использовать алгоритм, определенный в~\ref{RANDMOD}.

Для создания одноразовых личных ключей при выработке ЭЦП 
может использоваться алгоритм генерации, 
определенный в~\ref{GENK.Alg}.
%
Ключом данного алгоритма является~$d$,
а синхропосылкой~--- хэш-значение подписываемого сообщения.
  
%Личные ключи~--- числа из множества~$\{1,2,\ldots,q-1\}$~--- 
%при генерации, как правило, строятся по случайным или псевдослучайным 
%двоичным словам. Используемое преобразование слов в числа не должно 
%увеличивать частоту появления одних ключей в ущерб другим.

Один и тот же ключ~$d$ разрешено использовать как в алгоритме выработки~ЭЦП 
(в том числе для генерации одноразового ключа), 
так и в алгоритме разбора токена ключа.
Использование~$d$ в других алгоритмах запрещено.

В информационных системах ключи представляются двоичными словами.
Для обеспечения совместимости рекомендуется представлять
личный ключ~$d$ словом~$\langle d\rangle_{2\ell}$,
а открытый ключ~$Q$~--- словом~$\langle Q\rangle_{4\ell}$.

\section{Функция хэширования}\label{COMMON.Hash}

В алгоритмах выработки и проверки ЭЦП используется функция 
хэширования~$h$, которая ставит в соответствие подписываемому или 
проверяемому сообщению~$X$ его хэш-значение~$h(X)$.
%
%Эта функция применяется также
%в алгоритме~\ref{GENK.Alg} для генерации одноразового ключа
%при выработке ЭЦП.

На уровне стойкости~$\ell$ должна использоваться функция~$h$,
значениями которой являются двоичные слова длины~$2\ell$.
Например, при~$\ell=128$ в качестве~$h$ можно 
выбрать функцию, заданную алгоритмом~\algname{belt-hash}.

Функция~$h$ должна быть алгоритмически определена в ТНПА.
Алгоритму хэширования в ТНПА должен быть назначен уникальный идентификатор. 
Кодовое представление~$\OID(h)$ этого идентификатора 
используется в алгоритмах ЭЦП. 

Идентификатор функции хэширования должен полностью определять ее действие.
Недопустимы ситуации, когда для описания действия, кроме идентификатора,
требуется указывать дополнительные параметры, например
начальное хэш-значение.

%В алгоритме генерации одноразового личного ключа также 
%используется функция хэширования со значениями длины~$2\ell$. 
%Эта функция также должна быть определена 
%в некотором~ТНПА.
%Она может совпадать с функцией, которая применяется
%для хэширования подписываемых сообщений, 
%либо отличаться от нее.

\section{Транспорт ключа}\label{COMMON.Transport}

В алгоритмах транспорта ключа используется заголовок~ключа.
Заголовок представляет собой слово~$I\in\{0,1\}^{128}$,
которое описывает открытые атрибуты транспортируемого ключа,
например данные об отправителе, получателе, назначении ключа.
%
Формат заголовка фиксируется в конкретной информационной системе.
%
Заголовок может передаваться вместе с токеном~ключа.
%
%При разборе токена заголовок используется как контрольная характеристика
%и поэтому изменение заголовка приведет к сигналу о нарушении целостности~$X$,
%даже если этого в действительности не произошло.
%
Если необходимости в передаче атрибутов ключей нет,
то могут использоваться постоянные заголовки, 
которые не требуется передавать.
По умолчанию~$I=0^{128}$.

Один и тот же ключ может транспортироваться одновременно 
нескольким сторонам. В этом случае отправитель должен
создать токены ключа для каждой из сторон.
%
Если стороны-получатели используют одинаковые 
параметры эллиптической кривой, 
то при создании токенов может использоваться
один и тот же одноразовый~личный ключ~$k$.

