\section{Генерация одноразового личного ключа}\label{GENK}

\subsection{Входные и выходные данные}

Входными данными алгоритма генерации одноразового 
личного ключа являются порядок~$q$ группы точек эллиптической кривой, 
$2^{2\ell-1}<q<2^{2\ell}$,
личный ключ $d\in\{1,2,\ldots,q-1\}$ 
и хэш-значение~$H=h(X)\in\{0,1\}^{2\ell}$ подписываемого сообщения~$X$
(см. шаг~1 алгоритма~\ref{SIGN.Sign}). 
%
Дополнительно используется идентификатор~$\OID(h)$.
%
По входным данным определяется число $n=\ell/64$
($n=2$, $3$ или $4$).
                                                        
Выходными данными алгоритма является одноразовый личный
ключ~$k\in\{1,2,\ldots,{q-1}\}$.

\subsection{Вспомогательные алгоритмы и переменные}

{\bf Алгоритм~\algname{belt-hash}}.
Используется алгоритм хэширования~\algname{belt-hash}, 
описанный в~\ref{PARAMS.Aux}.

{\bf Алгоритм~\algname{belt-block}}.
Используется алгоритм зашифрования~\algname{belt-block}, 
определенный в СТБ~34.101.31 (шифрование блока, пункт~6.1.3).
Входными данными алгоритма являются слово~$X\in\{0,1\}^{128}$
и ключ~$\theta\in\{0,1\}^{256}$,
выходными~--- зашифрованное слово~$Y\in\{0,1\}^{128}$.

{\bf Переменная $r$}.
Используется переменная~$r=r_1\parallel r_2\parallel\ldots\parallel r_n$,
где $r_i\in\{0,1\}^{128}$.
%
Значение~$r$ должно быть уничтожено после использования.

{\bf Переменная $s$}.
Используется переменная~$s\in\{0,1\}^{128}$.
Значение~$s$ должно быть уничтожено после использования.

{\bf Переменная $t$}.
Используется переменная~$t\in\{0,1\}^*$.
Значение~$t$ выбирается произвольным, в том числе случайным
или псевдослучайным, образом. Может использоваться
фиксированное значение~$t$. По умолчанию $t$ полагается
пустым словом.
                                               
{\bf Переменная~$\theta$}.
Используется переменная~$\theta\in\{0,1\}^{256}$.
Значение~$\theta$ должно быть уничтожено после использования.

\subsection{Алгоритм генерации одноразового личного ключа}
\label{GENK.Alg}

Генерация одноразового~личного ключа 
состоит в выполнении следующих шагов:
\begin{enumerate}
\item
Выбрать~$t$ произвольным образом.
\item
Установить~$\theta\leftarrow\algname{belt-hash}
(\OID(h)\parallel\langle d\rangle_{2\ell}\parallel t)$.
\item
Установить $r\leftarrow H$.
\item
Для $i=1,2,\ldots$ выполнить:
\begin{enumerate}
\item
если $n=2$, то
\begin{enumerate}
\item
$s\leftarrow r_1$;
\end{enumerate}
\item
если $n=3$, то
\begin{enumerate}
\item
$s\leftarrow r_1\oplus r_2$;
\item
$r_1\leftarrow r_2$;
\end{enumerate}
\item
если $n=4$, то
\begin{enumerate}
\item
$s\leftarrow r_1\oplus r_2\oplus r_3$;
\item
$r_1\leftarrow r_2$;
\item
$r_2\leftarrow r_3$;
\end{enumerate}
\item
$r_{n-1}\leftarrow
\algname{belt-block}(s,\theta)\oplus r_n\oplus\langle i\rangle_{128}$;
\item
$r_n\leftarrow s$;
\item
если $i$ кратно~$2n$ и $\overline{r}\in\{1,2,\ldots,q-1\}$,
то перейти к шагу~5.
\end{enumerate}
\item
Установить~$k\leftarrow \overline{r}$.
\item
Возвратить~$k$.
\end{enumerate}

