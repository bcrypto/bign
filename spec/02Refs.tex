\chapter{Нормативные ссылки}

В настоящем cтандарте использованы ссылки на следующие 
технические нормативные правовые акты в области 
технического нормирования и стандартизации (далее~--- ТНПА):

СТБ~34.101.17-2012 Информационные технологии и безопасность.
Синтаксис запроса на получение сертификата

СТБ~34.101.19-2012 Информационные технологии и безопасность. 
Форматы сертификатов и списков отозванных сертификатов 
инфраструктуры открытых ключей

СТБ~34.101.23-2012 Информационные технологии и безопасность. 
Синтаксис криптографических сообщений 

СТБ~34.101.26-2012 Информационные технологии и безопасность. 
Онлайновый протокол проверки статуса сертификата (OCSP)

СТБ~34.101.31-2011 Информационные технологии. Защита информации.
Криптографические алгоритмы шифрования и контроля целостности

СТБ~34.101.47-2012 Информационные технологии и безопасность. 
Криптографические алгоритмы генерации псевдослучайных чисел

СТБ~1176.2-99 Информационная технология. Защита информации. 
Процедуры выработки и проверки электронной цифровой подписи

ГОСТ~34.973-91 (ИСО 8824-87) Информационная технология. Взаимосвязь
открытых систем. Спецификация абстрактно-синтаксической нотации
версии 1 (АСН.1)

ГОСТ 34.974-91 (ИСО 8825-87) Информационная технология. Взаимосвязь
открытых систем. Описание базовых правил кодирования для 
абстрактно-синтаксической нотации версии 1 (АСН.1)

\begin{note}
Примечание~---~При пользовании настоящим стандартом
целесообразно проверить действие ТНПА по каталогу, 
составленному по состоянию на 1 января текущего
года, и по соответствующим информационным указателям, опубликованным
в текущем году.
%
Если ссылочные ТНПА заменены (изменены), то при
пользовании настоящим стандартом следует руководствоваться
замененными (измененными) ТНПА. Если ссылочные ТНПА отменены без
замены, то положение, в котором дана ссылка на них, применяется в
части, не затрагивающей эту ссылку.
\end{note}

