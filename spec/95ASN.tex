\begin{appendix}{Д}{рекомендуемое}
{Модуль АСН.1}\label{ASN}

\hiddensection{Идентификаторы}\label{ASN.OIDs}

Алгоритмам стандарта присваиваются следующие идентификаторы:
\begin{center}
\begin{tabular}{p{4.2cm}p{11.8cm}}
\algname{bign-with-hspec} &
алгоритмы ЭЦП (\ref{SIGN}) c функцией хэширования, 
определяемой долговременными параметрами;\\
%
\algname{bign-with-hbelt} &
алгоритмы ЭЦП (\ref{SIGN}) c функцией хэширования,
заданной алгоритмом~\algname{belt-hash};\\
%
\algname{bign-genec} &
алгоритм генерации параметров эллиптической кривой (\ref{GENEC});\\
%
\algname{bign-valec} &
алгоритм проверки параметров эллиптической кривой (\ref{VALEC});\\
%
\algname{bign-genkeypair} &
алгоритм генерации пары ключей (\ref{GENKEYPAIR});\\
%
\algname{bign-valpubkey} &
алгоритм проверки открытого ключа (\ref{VALPUBKEY});\\
%
\algname{bign-keytransport} &
алгоритмы транспорта ключа (\ref{TRANSPORT});\\
%
\algname{bign-genk} &
алгоритм генерации одноразового личного ключа (\ref{GENK.Alg});\\
%
\algname{bign-ibs-with-hspec} &
алгоритмы ИЭЦП (\ref{IBS.Algs}) 
c функцией хэширования, определяемой долговременными параметрами;\\
%
\algname{bign-ibs-with-hbelt} &
алгоритмы ИЭЦП (\ref{IBS.Algs}) 
c функцией хэширования, заданной алгоритмом~\algname{belt-hash}.\\
\end{tabular}
\end{center}

Уровень стойкости алгоритмов ЭЦП, транспорта ключа и~ИЭЦП
не указывается в их идентификаторах, а
определяется по размерностям параметров используемой эллиптической кривой. 
%
Размерности параметров и длина значений используемой
функции хэширования должны соответствовать друг другу.

Открытому ключу, который вырабатывается по алгоритму~\ref{GENKEYPAIR}, 
присваивается идентификатор \algname{bign-pubkey}.
Открытый ключ может использоваться в алгоритмах ЭЦП 
и (или) транспорта ключа.

Стандартным параметрам эллиптической кривой,
заданным в приложении~\ref{STD}, присваиваются следующие идентификаторы:
\begin{center}
\begin{tabular}{p{4.2cm}p{11.8cm}}
\algname{bign-curve256v1} &
параметры, определенные в таблице~\ref{Table.STD.StdCurve128};\\
%
\algname{bign-curve384v1} &
параметры, определенные в таблице~\ref{Table.STD.StdCurve192};\\
%
\algname{bign-curve512v1} &
параметры, определенные в таблице~\ref{Table.STD.StdCurve256}.\\
\end{tabular}
\end{center}

\hiddensection{Параметры алгоритмов}\label{ASN.Params}

Идентификаторы алгоритмов ЭЦП, транспорта ключа и~ИЭЦП могут 
указываться в компоненте~\algname{algorithm} следующего типа~АСН.1:
\begin{verbatim}
AlgorithmIdentifier ::= SEQUENCE {
  algorithm   OBJECT IDENTIFIER,
  parameters  ANY DEFINED BY algorithm OPTIONAL
}
\end{verbatim}

Этот тип описывает, например, 
компоненты с именами~\algname{signatureAlgorithm} 
и~\algname{signature} в определениях форматов
запроса на получение сертификата (СТБ~34.101.17),
сертификатов и списка отозванных сертификатов (СТБ~34.101.19), 
подписанных данных (СТБ~34.101.23),
запроса и ответа о статусе сертификата (СТБ~34.101.26).
%
Во всех перечисленных случаях, если в~\algname{algorithm} установлен
идентификатор~\algname{bign-with-hspec}, то в~\algname{parameters} 
должен быть задан идентификатор используемой функции хэширования~$h$. 
%
Если же в~\algname{algorithm} установлен 
идентификатор~\algname{bign-with-hbelt},
то~\algname{parameters} должен равняться~\algname{NULL},
а~$h$ определяться алгоритмом~\algname{belt-hash}.

Тип~\algname{AlgorithmIdentifier} описывает также компоненты 
с именами \algname{keyEncryptionAlgorithm} в определениях форматов
конвертованных, аутентифицируемых и аутентифицируемых 
конвертованных данных (СТБ~34.101.23). 
Эти компоненты определяют способ защиты
ключей шифрования и имитозащиты данных. 
%
Если для защиты используются алгоритмы транспорта ключа
[в \algname{RecipientInfo} выбран компонент \algname{ktri}, 
см. СТБ~34.101.23 (пункт 9.3)],
а в~\algname{algorithm} установлен идентификатор~\algname{bign-keytransport},
то~\algname{parameters} должен равняться~\algname{NULL},
а заголовок~$I$ транспортируемого ключа полагаться равным~$0^{128}$.

\hiddensection{Описание двоичных слов}\label{ASN.Words}

Слово~$u\in\{0,1\}^{8*}$ может описываться строкой битов
(\algname{BIT STRING}), последовательными 
[от первого к замыкающему, см. ГОСТ 34.973 (раздел~15)]
элементами которой являются последовательные символы~$u$.
%
Слово~$u$ может также описываться строкой октетов 
(\algname{OCTET STRING}), составленной из последовательных 
октетов~$u$.

Указанные описания~$u$ согласованы между собой. 
Например, при кодировании~$u$ по правилам ГОСТ 34.974 
содержимое~$u$ как \algname{BIT STRING} отличается
от содержимого~$u$ как~\algname{OCTET~STRING} 
только нулевым октетом-префиксом.

\hiddensection{Описание конечного поля и его элементов}\label{ASN.Field}

Для описания конечного поля, над которым строится эллиптическая кривая, 
используется тип
\begin{verbatim}
FieldID ::= SEQUENCE {
  fieldType   OBJECT IDENTIFIER (bign-primefield),
  parameters  INTEGER
}
\end{verbatim}

Компонент~\algname{fieldType} этого типа определяет вид поля.
Примененный синтаксис обязывает использовать только простые 
конечные поля, которым назначен идентификатор~\algname{bign-primefield}.
Компонент~\algname{parameters} описывает модуль~$p$ поля~$\FF_p$.

На уровне стойкости~$\ell$
элемент~$u$ поля~$\FF_p$ должен представляться строкой 
октетов как двоичное слово~$\langle u\rangle_{2\ell}$.

\hiddensection{Описание параметров эллиптической кривой}\label{ASN.EC}

Параметры эллиптической кривой 
могут представляться значениями типа
\begin{verbatim}
DomainParameters ::= CHOICE {
  specified  ECParameters,
  named      OBJECT IDENTIFIER,
  implicit   NULL 
}
\end{verbatim}

Выбор компонента~\algname{specified} означает явное задание параметров. 
%
Компонент~\algname{named} используется для ссылки на именованные параметры, 
заданные в приложении~\ref{STD} или в другом документе.
%
Компонент~\algname{implicit} используется для указания на то, что наследуются
параметры внешнего источника, например удостоверяющего центра.

Явно задаваемые параметры представляются значениями типа
\begin{verbatim}
ECParameters ::= SEQUENCE {
  version  INTEGER {ecpVer1(1)} (ecpVer1),
  fieldID  FieldID,
  curve    Curve,
  base     OCTET STRING (SIZE(32|48|64)),
  order    INTEGER,
  cofactor INTEGER (1) OPTIONAL
}
\end{verbatim}

Компонент~\algname{version} указывает на версию данного типа АСН.1.
Примененный синтаксис обязывает использовать версию~$1$,
которая обозначена через~\algname{ecpVer1}.
%
Компонент~\algname{fieldID} описывает поле~$\FF_p$, 
над которым строится эллиптическая кривая.
%
Компонент~\algname{curve} описывает уравнение эллиптической кривой.
%
Компонент~\algname{base} описывает базовую точку эллиптической кривой.
%
Компонент~\algname{order} описывает порядок~$q$ 
группы точек эллиптической кривой.
%
Необязательный компонент~\algname{cofactor} описывает отношение
порядка всей группы точек эллиптической кривой к порядку группы, 
порожденной базовой точкой.                        
Это отношение (кофактор) для эллиптических кривых
настоящего стандарта всегда равняется~$1$.

Для описания уравнения эллиптической кривой используется тип
\begin{verbatim}
Curve ::= SEQUENCE {
  a     OCTET STRING (SIZE(32|48|64)),
  b     OCTET STRING (SIZE(32|48|64)),
  seed  BIT STRING (SIZE(64))
}
\end{verbatim}

Последовательные компоненты этого типа определяют 
коэффициенты~$a$, $b$ эллиптической кривой и параметр~$seed$,
использованный для построения~$b$ при заданных~$p$ и~$a$.

Коэффициенты~$a$, $b$ и базовая точка~$G=(0,y_G)$ 
задаются строками октетов.
%
Эти строки строятся по $a$, $b$ и $y_G$ как элементам поля~$\FF_p$
по правилам, определенным в~\ref{ASN.Field}. 
Нулевая $x$-координата точки~$G$ опускается.

\hiddensection{Описание открытого ключа}\label{ASN.PubKey}

На уровне стойкости~$\ell$ открытому ключу~$Q$ 
ставится в соответствие двоичное слово~$\langle Q\rangle_{4\ell}$,
для описания которого может использоваться тип
\begin{verbatim}
PublicKey ::= BIT STRING (SIZE(512|768|1024))
\end{verbatim}

В запросе на получение сертификата (СТБ~34.101.17) 
и в сертификатах (СТБ~34.101.19)
открытый ключ должен представляться значениями типа
\begin{verbatim}
SubjectPublicKeyInfo ::= SEQUENCE {
  algorithm         AlgorithmIdentifier,
  subjectPublicKey  PublicKey
}
\end{verbatim}

Компонент~\algname{algorithm} этого типа 
описывает свойства открытого ключа.
Компонент~\algname{subjectPublicKey} 
описывает значение открытого ключа.

Для описания свойств открытого ключа в компоненте
\algname{algorithm}, вложенном в~\algname{AlgorithmIdentifier},
должен быть установлен идентификатор~\algname{bign-pubkey},
а в компоненте~\algname{parameters}~--- значение 
типа~\algname{DomainParameters}.

%\hiddensection{Описание личного ключа}
%
%Личный ключ алгоритмов ЭЦП и транспорта ключа 
%может представляться значениями типа
%\begin{verbatim}
%ECPrivateKey ::= SEQUENCE {
%  version     INTEGER {ecpkVer1(1)} (ecpkVer1),
%  privateKey  OCTET STRING (SIZE(32|48|64)),
%  parameters  DomainParameters OPTIONAL,
%  publicKey   BIT STRING (SIZE(512|768|1024)) OPTIONAL
%}
%\end{verbatim}
%
%Компонент~\algname{version} указывает на версию данного типа АСН.1.
%Примененный синтаксис обязывает использовать версию~$1$,
%которая обозначена через~\algname{ecpkVer1}.
%%
%Компонент~\algname{privateKey} описывает значение личного ключа.
%Необязательные компоненты \algname{parameters}, \algname{publicKey}
%определяются так же, как в~\ref{ASN.PubKey}.
%
%На уровне стойкости~$\ell$ личному ключу~$d$ ставится в соответствие 
%двоичное слово~$\langle d\rangle_{2\ell}$,
%которое разбивается на~$\ell/4$ последовательных октетов.
%Полученная строка октетов определяет значение
%компонента~\algname{privateKey}.

%Тип \algname{PrivateKey} может быть использован для описания личного 
%ключа при помощи синтаксиса PKCS\#8. При этом должен использоваться тип 
%\algname{PrivateKeyInfo} и компонент \algname{privateKeyAlgorithm} этого
%типа должен принимать значение \algname{bign-pubkey}.

\hiddensection{Описание электронной цифровой подписи}
\label{ASN.Sign}

На уровне стойкости~$\ell$ ЭЦП является двоичным словом длины~$3\ell$,
для описания которого могут использоваться типы
\begin{verbatim}
Signature ::= BIT STRING (SIZE(384|576|768))
SignatureValue ::= OCTET STRING (SIZE(48|72|96))
\end{verbatim}

В запросе на получение сертификата (СТБ~34.101.17),
в сертификатах и списках отозванных сертификатов (СТБ~34.101.19),
в запросе и ответе о статусе сертификата (СТБ~34.101.26)
подпись должна представляться значением типа~\algname{Signature}, 
а в подписанных данных (СТБ~34.101.23)~---
значением типа~\algname{SignatureValue}.

\hiddensection{Описание токена ключа}\label{ASN.Token}

В алгоритмах транспорта \algname{bign-keytransport}
токен ключа представляет собой двоичное слово, 
длина которого кратна~$8$.

В  определениях форматов конвертованных, аутентифицируемых 
и аутентифицируемых конвертованных данных (СТБ~34.101.23)
токен ключа должен устанавливаться 
в компоненте~\algname{encryptedKey} типа~\algname{KeyTransRecipientInfo}
и описываться при этом строкой октетов.

\hiddensection{Описание идентификационной электронной цифровой подписи}
\label{ASN.IBS}

На уровне стойкости~$\ell$ объединение ИЭЦП с открытым ключом,
описанное в~\ref{IBS.Format}, является двоичным словом длины~$7\ell$.
Для описания этого слова может использоваться тип
\begin{verbatim}
IdSignatureValue ::= OCTET STRING (SIZE(112|168|224))
\end{verbatim}

\hiddensection{Совместимость}\label{ASN.Compat}

Введенные определения поддерживают использование алгоритмов 
настоящего стандарта при создании и обработке данных, 
форматы которых определены в СТБ~34.101.17, СТБ~34.101.19, 
СТБ~34.101.23, СТБ~34.101.26.

Тип~\algname{ECParameters} соответствует одноименному типу,
определенному в~\cite{ANSI9.62}, со следующими уточнениями:
\begin{enumerate}
\item
Простое конечное поле описывается идентификатором~\algname{bign-primefield} 
вместо  идентификатора \algname{\{iso(1) member-body(2) us(840) 10045 
4 1 field-type(1) 1\}} в~\cite{ANSI9.62}.

\item
Элементы поля описываются строками октетов по правилам
<<от младших к старшим>> (little-endian), 
принятым в настоящем стандарте, 
вместо правил <<от старших к младшим>> (big-endian), 
принятым в~\cite{ANSI9.62}.

\item
В компоненте~\algname{base} задается только $y$-координата
базовой точки эллиптической кривой. Нулевая $x$-координата опускается.
\end{enumerate}

Тип~\algname{DomainParameters} соответствует типу~\algname{Parameters}, 
определенному в~\cite{ANSI9.62}.

%Тип~\algname{DomainParameters} соответствует типу~\algname{Parameters} 
%стандарта~\cite{ANSI9.62} и типу~\algname{ECDomainParameters} 
%стандарта~\cite{SEC1}.

%Тип~\algname{ECPrivateKey} соответствует одноименному типу 
%стандарта~\cite{SEC1}.

\hiddensection{Модуль АСН.1}\label{ASN.Module}

\verbatiminput{bign-module-v2.asn}

\end{appendix}
